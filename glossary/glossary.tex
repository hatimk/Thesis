\chapter*{Glossary}

\paragraph{Automatic speech recognition (ASR):} Module transforming the user's speech into an N-Best.

\paragraph{Client:} Interface between the user and the system. In the case of spoken dialogue systems, it contains the ASR and the TTS.

\paragraph{Confidence score:} Positive real number associated with an ASR or a NLU output hypothesis. The higher it is, the more likely this hypothesis is the correct one.

\paragraph{Dialogue manager (DM):} Module computing the system's response in the form of concepts given a user's request.

\paragraph{Dialogue system:} System communicating with users through natural interaction using natrual language.

\paragraph{Incremental dialogue system:} Dialogue systems which are able to process the user's speech as it is spoken, before the end of the sentence. They are able to take the floor while the user is still speaking and inversely, the latter can also do so while the system is speaking.

\paragraph{Micro-turn:} Time interval between two updates of an incremental dialogue system.

\paragraph{N-Best:} N ASR or NLU hypotheses that are associated with the N best confidence scores.

\paragraph{Natural language generation (NLG):} Module transforming the concepts computed by the DM into text.

\paragraph{Natural language understanding (NLU):} Module transforming the user's sentences in text format into concepts understandable by the DM.

\paragraph{Reinforcement learning:} Machine learning framework where the learning agent learns by trial and error while directly interacting with an environment.

\paragraph{Scheduler:} Turn-taking management module inserted between the Client and the Service. The set \{Scheduler+Service\} behaves like an incremental dialogue system from the Client's point of view.

\paragraph{Service:} Backend part of a dialogue system. It contains the DM, it manages the backend I/O (database access, etc.) and depending on the architecture, it may also contain the NLU and/or the NLG.

\paragraph{Spoken dialogue system (SDS):} Dialogue system aimed to communicate through speech. It uses an ASR and a TTS.

\paragraph{Turn-taking phenomenon (TTP):} Behaviours and mechanisms through which the conversation participants take turns in speaking during a conversation.

\paragraph{Text-to-speech (TTS):} Module transforming text into a vocal signal. Also called speech synthesis.

\paragraph{Turn-taking:} The act of taking the floor during a conversation.

\paragraph{User simulator (US):} System aimed to simulate user behaviours. It makes it possible to quickly generate important dialogue corpora, which is especially interesting when building machine learning algorithms.

\chapter{Dialogue strategies}
\label{ch:strategies}

\section{Utterance representation}
\label{sec:represent}
	
	When interacting with a dialogue system, the user's utterance can be represented in different ways. A simple way of representing the different chunks of information that it contains is slot-filling. For example, if the user says \textit{I would like to buy the book entitled Dracula and written by Bram Stoker}, the NLU can output the following matrix:
	
		$$
		\begin{bmatrix}
			\text{ACTION:} & \text{Buy} \\
			\text{OBJECT:} & \text{Book} \\
			\text{TITLE:} & \text{Dracula} \\
			\text{WRITER:} & \text{Bram Stoker}
		\end{bmatrix}
		$$
	
	Each entry is called a \textit{slot}, it has two attributes: the \textit{slot name} (on the left) and the \textit{slot value} (on the right). Of course, depending on the application at hand, this matrix could be different; some slots could be considered as non relevant hence being discarded, others could be combined and new slots could be added. Therefore, while designing a dialogue system, a representation is chosen and one has to stick to it. This representation is widely used when it comes to dialogue systems design since it is a simple and natural way of representing the user's utterance. It is well adapted to most tasks where the dialogue system plays the role of an interface between a user and a data base. The system expects a user's request with predefined slots and depending on the latter and the data base content, a response is computed and returned. Therefore, this is the representation used in this thesis and Section \ref{sec:elemtask} provides more precisions about this topic.

\section{Elementary tasks}
\label{sec:elemtask}

	A \textit{task} is associated with an objective that the user wants to achieve. For example:
	
	\begin{itemize}
		\item Checking a bank account
		\item Scheduling an appointment
		\item Finding a restaurant nearby
		\item Booking a hotel room
		\item Modifying a flight reservation
	\end{itemize}
	
	Some tasks involve retrieving information from a database and others make the system modify the outside world (data base modification, robot movement...). Moreover, a task can involve several \textit{elementary tasks}. An elementary task is defined as an atomic action performed by the system after it the user has made a request with all the necessary information. The task of driving a robot from point A to point B could involve the following elementary tasks.
	
	\begin{enumerate}
		\item Go straightforward until you reach the STOP sign
		\item Turn right
		\item Go straightforward until you reach the wall
		\item turn left
		\item Go straightforward until you reach the point B
	\end{enumerate}
	
	The way a task is organised in elementary tasks might affect the efficiency of its execution. However, this is not under the control of the dialogue system. The latter only controls the way elementary tasks are handled. As a consequence, this thesis focuses on the optimisation of elementary tasks using different dialogue strategies.
	
	To complete an elementary task, the system requires information that can be represented as a slot matrix:
	
		$$
		\begin{bmatrix}
			s_1 & x_1 \\
			s_2 & x_2 \\
			... & ... \\
			s_n & x_n
		\end{bmatrix}
		$$
	
	In this framework, a dialogue system is viewed as an automaton that is able to perform different types of elementary tasks. Hence, a dialogue is a sequence of elementary tasks: $ET_1,...,ET_k,...$. $ET_k$ consists on performing an action $a_k$ as a response to a user's request $X_k = (x_1, x_2, ...)$ which a vector containing the slot values provided in his utterance (the size of the vector depends on the type of elementary task). Let $C_k$ be the dialogue context when it is time to compute $a_k$, then $a_k = f(X_k,C_k)$, $f$ being defined in the dialogue system.
	
	Moreover, a distinction can be made between two kinds of slots:
	
	\begin{itemize}
		\item \textbf{Open slots:} Every input from the user can be considered as a valid value. For example, if the user is asked to utter a message that will be sent to some of his friends later on, he can utter anything. Therefore, unless the system asks for a specific confirmation, it cannot determine whether the user's input is right or wrong.
		\item \textbf{Constrained slots:} Some user's inputs are considered valid (but not necessarily correct) and others are not. If the slot at hand is a date and the user responds something which is not a date, this response is not valid in which case is sure that the slot value is wrong. However, the validity of the response does not guarantee its correctness but it is noteworthy that if a generic ASR with open grammar is used, valid responses that are incorrect are quite rare. On the other hand, if the ASR uses a grammar that is specific to the task at hand, they are more likely to happen.
	\end{itemize}
	
	The way these slots are communicated depends on the dialogue strategy, the way the requests are formulated, the noise level and the NLU's ability to recognise a large variety of words and expressions. In Section \ref{sec:slotfillstrat}, a different slot-filling strategies are introduced and discussed.

\section{Slot-filling strategies}
\label{sec:slotfillstrat}

	Depending on the task, the dialogue system and the user, slots can be filled in different ways in order to complete an elementary task. Three generic strategies are presented here but before delving into that, the notion of \textit{initiative} is introduced. Consider the following dialogue between a customer and a receptionist:
	
	\begin{dialogue}
		\speak{Customer} Hi. I would like to book a room please, leaving on Monday.
		\speak{Receptionist} Sure. Are you driving sir?
		\speak{Customer} No. I came here by train.
		\speak{Receptionist} All right. Would you like a smoking room?
		\speak{Customer} No, I don't smoke.
		\speak{Receptionist} Ok. What about breakfast?
		\speak{Customer} Yes, please. I will have it here.
		\speak{Receptionist} Great, all set then! Let me get your key...
		\speak{Customer} Thanks. When should I check out on Monday?
		\speak{Receptionist} Checkout is before 11:30am.
		\speak{Customer} Ok. My train is leaving at 6:00pm, would it be possible to leave my bag here and get it back by then?
		\speak{Receptionist} Absolutely sir. No problem.
	\end{dialogue}

	This dialogue can be split into three phases. First, the customer starts the conversation by making a request. It is a result of his own initiative and he is setting the subject of the conversation. Then the receptionist provides an answer and takes the initiative right after by starting to ask specific questions. When, all the necessary information for the reservation is provided, the customer takes the initiative again to get some additional clarifications.
	
	Such a distinction can also be made in the case of dialogue systems \cite{Ferguson2007}. When using a \textit{system initiative} strategy, the latter makes requests and asks questions that the user should respond to in order to move the dialogue forward. Symmetrical strategies are called \textit{user initiative} (the user leads the course of the dialogue). Finally, strategies that involve both dialogue modes are called \textit{mixed initiative} strategies. This three dialogue strategies are presented and discussed in the following.
	
	\subsection{System initiative strategies}
	
		System initiative strategies can be compared to form-filling. The user is asked to fill several slots in a progressive way. Slot values can be asked for one by one or little group by little group. However, the \textit{one by one} case is the most common and the most simple, therefore, this thesis will focus on that case.
		
		Formally, consider an elementary task $ET$ that involves $n$ slots: $s_1, ..., s_n$. Completing $ET$ using the system initiative strategies consists on a dialogue with $n$ system questions $q(s_1),...,q(s_n)$ followed by the $n$ corresponding answers containing the required slot values $x_1,...,x_n$. If an error occurs when trying to get $x_k$, the problem is reported and the user is asked to repeat (or reformulate) her response again. To recap, a user initiative dialogue looks like the following:
		
		\begin{dialogue}
			\speak{SYSTEM} $q(s_1)$
			\speak{USER} $x_1$
			\speak{SYSTEM} $q(s_2)$
			\speak{USER} <noise>
			\speak{SYSTEM} Sorry, I don't understand. $q(s_2)$
			\speak{USER} $x_2$
			[...]
			\speak{SYSTEM} $q(s_n)$
			\speak{USER} $x_n$
			\speak{SYSTEM} Confirmation message.
		\end{dialogue}
		
		A misunderstanding can happen for two reasons: either the system does not understand anything because or noise, either the ASR outputs an utterance but it is not compatible with the type of information required. The latter situation happens only in the case of constrained slots. Therefore, the reader might notice that it is very easy for the system to accept an erroneous answer in the case of an open slot as any answer, except empty ones (because of noise), is accepted by the system. The problem becomes visible by the user at the confirmation phase only making him restart the current elementary task from the beginning. Since this inevitably leads to a very tedious dialogue strategy, in this thesis, intermediate confirmations are added after each open slot response in order to be able to correct them immediately before tackling the remaining slots.
	
	\subsection{User initiative strategies}
	
		When the user initiative strategy is used to complete an elementary task $ET$ (involving $n$ slots $s_1,...,s_n$), the user is supposed to provide all the slot values in a complete utterance. If there are missing slots in his request, he is asked to repeat (or reformulate) it. The dialogue then looks as follows:
		
		\begin{dialogue}
			\speak{SYSTEM} What can I do for you?
			\speak{USER} $x_1$, <noise>,...,$x_n$
			\speak{SYSTEM} Confirmation message.
			\speak{USER} No.
			\speak{SYSTEM} What can I do for you?
			\speak{USER} $x_1$, $x_2$,...,$x_n$
			\speak{SYSTEM} Confirmation message.
		\end{dialogue}
	
	\subsection{Mixed initiative strategies}
	
		In noisy environments, the user initiative strategy can be very tiring to the user, especially when the number of slots is important. Another way to deal with incomplete requests is to switch to the system initiative strategy to gather the missing slots. Suppose that the elementary task at hand $ET$ involves $n=5$ slots: $s_1,...,s_5$. A mixed initiative strategy dialogue looks like the following:
		
		\begin{dialogue}
			\speak{SYSTEM} What can I do for you?
			\speak{USER} $x_1$, <noise>,$x_3$,<noise>,$x_n$
			\speak{SYSTEM} $q(s_2)$
			\speak{USER} $x_2$
			\speak{SYSTEM} $q(s_4)$
			\speak{USER} $x_4$
			\speak{SYSTEM} Confirmation message.
		\end{dialogue}
	
	\subsection{First efficiency comparison}
	
		

\section{Incremental strategies}
\label{sec:incrstrat}
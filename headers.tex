%\documentclass[a4paper,8pt,twocolumn]{report}
\documentclass[a4paper,11pt]{book}
%\documentclass[a4paper,9pt]{report}
\usepackage[english]{babel}
%\usepackage[frenchb]{babel}
\usepackage[T1]{fontenc}           % for hyphenations to work
\usepackage[latin1]{inputenc}
\usepackage{amsmath,amssymb,amsfonts,textcomp}
\usepackage{color}
\usepackage{graphicx}
%\usepackage[french,figure]{algorithm2e}
\usepackage{tabularx}
%\usepackage[frenchb]{minitoc}
\usepackage{minitoc}
\usepackage{multirow}
\usepackage[round,authoryear]{natbib}
\usepackage{fancyhdr}
\usepackage{afterpage}
\usepackage{longtable}

%\renewcommand{\chaptermark}[1]{\markboth{\chaptername\ \thechapter. #1}{}}
%\renewcommand{\sectionmark}[1]{\markright{\thesection. #1}}
%% Setting up pagestyles for ``fancy''
\chead{}
\rhead[]{\nouppercase{\rightmark}}
\lhead[\nouppercase{\leftmark}]{}
\renewcommand{\headrulewidth}{0.5pt}
%\footrulewidth 0.5pt
\lfoot{}
\cfoot{\thepage}
\rfoot{}

% watch a movie with one frame on each page
%\rfoot{\setlength{\unitlength}{1mm} 
%\begin{picture}(0,0) 
%\put(5,0){\includegraphics{pic\thepage.ps}} 
%\end{picture}} 

% �quivalent � \usepackage{palatino}
\usepackage{mathpazo} 
\usepackage[scaled=.95]{helvet} 
\usepackage{courier} 

\usepackage[twoside]{geometry}
\setlength{\oddsidemargin}{1cm}
\setlength{\evensidemargin}{0cm}
\addtolength{\textheight}{1cm}
\addtolength{\headsep}{0.3cm}
\addtolength{\footskip}{0.3cm}
\usepackage{layout}
%\newlength{\oldtextwidth}
%\setlength{\oldtextwidth}{\textwidth}
%\usepackage[margin=1cm]{geometry}
%\usepackage[margin=2.5cm]{geometry}
\usepackage{setspace}

\usepackage{boxedminipage}
\usepackage{rotating}
\newcommand\vertical[1]{\begin{sideways}#1\end{sideways}}
\usepackage[width=.95\textwidth,font={small,it},labelfont=bf]{caption}
%\renewcommand{\captionfont}{\small\itshape}
\setlength{\fboxsep}{5pt}

\usepackage{parskip}
\setlength{\parindent}{1.5em}
% re-d�finition des marges
%\textwidth = 15 cm
%\textheight = 22 cm
%\oddsidemargin = 1.0 cm
%\evensidemargin = 0.0 cm
%\topmargin = 0.0 cm
%\headheight = 0.0 cm
%\headsep = 1.0 cm
%\setlength{\parskip}{0.5em}
%\parindent = 0.6 cm
%\abovecaptionskip = 0 cm

% line badness cooking
%\tolerance 1414 
%\hbadness 1414 
%\emergencystretch 1.5em 
%\hfuzz 0.3pt 
%\widowpenalty=10000 
%\vfuzz \hfuzz 
%\raggedbottom 

%\prefacename Preface Pr�face 
%\refname a References R�f�rences 
%\abstractname Abstract R�sum� 
%\bibname b Bibliography Bibliographie 
%\chaptername Chapter Chapitre 
%\appendixname Appendix Annexe 
%\contentsname Contents Table des mati�res 
%\listfigurename List of Figures Table des figures 
%\listtablename List of Tables Liste des tableaux 
%\indexname Index Index 
%\figurename Figure F I G. 
%\tablename Table TAB . 
%\partname Part partie 
%\enclname encl P. J. 
%\ccname cc Copie � 
%\headtoname To 
%\pagename Page page 
%\seename see voir 
%\alsoname see also voir aussi 
%\addto{\captionsfrench}{\renewcommand*{\listfigurename}{Liste des illustrations}} 

%\renewcommand\floatpagefraction{.9}
%\renewcommand\topfraction{.9}
%\renewcommand\bottomfraction{.9}
%\renewcommand\textfraction{.1}   
%\setcounter{totalnumber}{50}
%\setcounter{topnumber}{50}
%\setcounter{bottomnumber}{50}

%\clubpenalty=9999
%\widowpenalty=9999

%\usepackage[plainpages=false,pdfpagelabels,hypertexnames=false]{hyperref}
\usepackage{hyperref}
%\hypersetup{plainpages=false,pdfpagelabels,hypertexnames=false,colorlinks=true, linkcolor=black, filecolor=black, pagecolor=black, urlcolor=black, citecolor=black}
\hypersetup{pagebackref=true,backref=true,plainpages=false,pdfpagelabels,hypertexnames=false,colorlinks=true, linkcolor=blue, filecolor=blue, pagecolor=blue, urlcolor=blue, citecolor=blue,pdftitle={Th\`{e}se},pdfauthor={Nom de l'auteur},pdfsubject={LaTex}}

%\newcommand\textsubscript[1]{\ensuremath{_{\text{#1}}}}
\everymath{\displaystyle}

%\newcommand\todo[1]{[\textcolor[rgb]{0,0,0}{\textbf{� faire:} \textit{#1}}]}
\newcommand\todo[1]{ [\textcolor[rgb]{0.5,0,0}{\textbf{� faire:} #1}] }
%\newcommand\todo[1]{}
%\newcommand\tocite[1]{[\textcolor[rgb]{0,0,0}{\textbf{� citer:} \textit{#1}}]}
\newcommand\tocite[1]{[\textcolor[rgb]{0,0.5,0}{\textbf{� citer:} #1}]}
%\newcommand\tocite[1]{}

%\renewcommand\label[1]{[\textcolor[rgb]{0,0,0.5}{\textbf{label:} #1\label{#1}}]}
\newcommand\quotes[1]{\guillemotleft\nobreak\hspace{1.5pt}#1\nobreak\hspace{1.5pt}\guillemotright}
\newcommand\cfigurex[4]
{
	\begin{figure}[hbt!]
	\centering
	\includegraphics[width=#4\textwidth]{#1}
	\caption[#2]{#3}
	\end{figure}
}

\newcommand\cfigure[3]
{
	\begin{figure}[hbt!]
	\centering
	\includegraphics[width=#3\textwidth]{#1}
	\caption{#2}
	\end{figure}
}

\newcommand\cfigurebox[3]
{
	\begin{figure}[hbt!]
	\centering
	\begin{boxedminipage}{#3\textwidth}
	\includegraphics[width=\textwidth]{#1}
	\end{boxedminipage}
	\caption{#2}
	\end{figure}
}

\newcommand\ctablex[4]%
{%
	\begin{table}[hbt]
	\centering
	\begin{tabular}{ #1}
	\hline
	#4
	\hline
	\end{tabular}
	\caption{\label{#3} #2}
	\end{table}
}

\newcommand\ctablexx[5]%
{%
	\begin{table}[hbt]
	\centering
	\begin{tabular}{ #1}
	\hline
	#5
	\hline
	\end{tabular}
	\caption[#2]{\label{#4} #3}
	\end{table}
}

\def\max{\mathop{\hbox{max}}}
\def\argmax{\mathop{\hbox{argmax}}}
\def\min{\mathop{\hbox{min}}}
\def\argmin{\mathop{\hbox{argmin}}}
\def\nb{\mathop{\hbox{nb}}}

%\newcommand{\citep}[1]{\cite{#1}}
%\newcommand{\citet}[1]{\cite{#1}}
\renewcommand{\cite}[1]{\citep{#1}}
%\newcommand{\refp}[1]{\ref{#1}, page \pageref{#1}}
%\newcommand{\refp}[1]{\ref{#1}}

%\newcommand\chapterx[1]{\chapter*{#1}\addcontentsline{toc}{chapter}{#1}}
%\newcommand\sectionx[1]{\section*{#1}\addcontentsline{toc}{section}{#1}}
%\newcommand\chapterx[1]{\chapter{#1}}
%\newcommand\sectionx[1]{\section{#1}}
%\newcommand\subsectionx[1]{\section{#1}}

%\newcommand{\etc}{, etc.}
\newcommand{\etc}{...}
%\setcounter{tocdepth}{5}
%\hyphenation{sur-ap-pren-tis-sage}
\pagestyle{plain}
\newcommand{\postheader}{
%\renewcommand{\minitoc}{}
%\doublespacing
%\addcontentsline{toc}{chapter}{Liste des tableaux}
\pagestyle{fancy}
%\bibliographyunit[\chapter] 
%\defaultbibliography{biblio-exact} 
%\defaultbibliographystyle{lia-these-fr} 
%\begin{bibunit}
}

%\usepackage{bibunits}
%\newcommand\chapterbib{}
%\usepackage[sectionbib]{bibunits}
%\newcommand\chapterbib{\putbib}

\newcommand\postdocument{
\cleardoublepage
\addstarredchapter{Liste des illustrations}
\listoffigures
\cleardoublepage
\addstarredchapter{Liste des tableaux}
\listoftables 
%\bibliographystyle{apalike-fr}
%\bibliographystyle{plain}
\cleardoublepage
\addstarredchapter{Bibliographie}

%\putbib[biblio-exact]
\bibliographystyle{lia-these-fr}
\bibliography{biblio}

%\end{bibunit}
%\begin{bibunit}
%\renewcommand*{\bibname}{Publications Personnelles}
%\addstarredchapter{Publications Personnelles}
%\nocite{*}
%\putbib[biblio-perso]
%\end{bibunit}


}

\usepackage{xcolor}
\usepackage{dialogue}
\usepackage{rotating}

\usepackage{tikz}
\usepackage{pgfplots}

\pgfplotsset{compat=1.9}
\usepgfplotslibrary{colorbrewer}

\usepackage[]{algorithm2e}
\usepackage{subcaption}

\usepackage{booktabs}
\newcommand{\tabitem}{~~\llap{\textbullet}~~}

\usepackage{multirow}

\usepackage{colortbl}
\usepackage{color}
\definecolor{dialogueInit}{rgb}{0.8, 0.8, 0.8}
\definecolor{negFb}{rgb}{1.0, 0.5, 0.5}
\definecolor{posFb}{rgb}{0.1, 0.5, 1.0}
\definecolor{ref}{rgb}{1.0, 1.0, 0.5}
\definecolor{orderedDA}{rgb}{0.5, 1.0, 0.5}

\definecolor{color1}{RGB}{228,26,28}
\definecolor{color2}{RGB}{55,126,184}
\definecolor{color3}{RGB}{77,175,74}
\definecolor{color4}{RGB}{152,78,163}
\definecolor{color5}{RGB}{255,127,0}
\definecolor{color6}{RGB}{255,255,51}
\definecolor{color7}{RGB}{166,86,40}
\definecolor{color8}{RGB}{247,129,191}
\definecolor{color9}{RGB}{153,153,153}
\definecolor{color10}{RGB}{158,10,54}
\definecolor{color11}{RGB}{100,50,100}

\usepackage{mathrsfs}
\chapter*{State of the art conclusion}

     The previous two chapters provide an overview of the current state of art related to dialogue systems in general, incremental dialogue systems as well as reinforcement learning and its applications to human-machine dialogue. In Chapter \ref{ch:soadialogue}, a few generic notions about human dialogue and the philosophy of language are discussed before providing a definition of what is a dialogue system as well as its different components. Incremental dialogue is also defined afterwards and the different advantages, challenges and architectural considerations that are associated with it are depicted. Reinforcement learning is then discussed in Chapter \ref{ch:soarl} and a state of the art relative to its application to dialogue systems is provided. Both traditional and incremental dialogue systems are covered.

     The next chapters are dedicated to the contributions that have been made during the course of this thesis. A turn-taking phenomena taxonomy in human dialogue has been proposed and used to build a rule-based turn-taking strategy. Then, a new architecture aimed at transforming a traditional dialogue system into an incremental one at a low cost has been built. Also, an incremental user simulator has been implemented and used to build an optimal turn-taking strategy using reinforcement learning, which has been shown to outperform its rule-based counterpart. Finally, these results have been validated in a live study with real users.
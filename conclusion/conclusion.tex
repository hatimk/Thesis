\chapter*{Conclusion}

	Viewed as a whole, the several contributions made in this thesis constitute a thorough methodology to enhance turn-taking capabilities of spoken dialogue systems. First, turn-taking mechanisms involved in human conversation are analysed which led to the establishment of a new turn-taking phenomena taxonomy. Compared to existing classifications in the literature, new analysis dimensions where the meaning behind the dialogue participants' behaviours, as well as the motivations behind them, are considered. This leads to a more fine-grained taxonomy which is also relevant from the human-machine dialogue point of view. In addition, this constitutes the starting point from which the turn-taking phenomena that are the more likely to improve the dialogue efficiency are chosen.

        An incremental dialogue system can be built from scratch using an incremental version of each component in the dialogue chain. In this thesis, an alternative approach is proposed: a dialogue system is split into a Client and a Service part, then a new module, called the Scheduler, is inserted between the two. This new interface plays the role of a turn-taking manager and makes the set \{Scheduler+Service\} behave as an incremental Service from the Client's point of view. Two advantages are associated with this new approach: first, it makes it possible to transform an existing traditional dialogue system into an incremental one at a low cost and second, the traditional dialogue management part is clearly separated from the turn-taking management one.

        Based on this new architecture, an incremental dialogue simulator has been implemented. It is able to generate dialogues in a personal agenda management domain. A User Simulator, coupled with an ASR Output Simulator that replicates ASR imprefections and instability, sends incremental requests to the Scheduler which decides when to take the floor to provide a response. A first simulation study where several slot-filling strategies along with two turn-taking strategies (incremental and handcrafted incremental) showed that the mixed-initiative one along with incremental processing achieves the best performance in terms of dialogue duration and task completion.

        Since hancrafting turn-taking strategies require the designers to empirically set all the parameters and doing so for each new task, this method is not guaranteed to be optimal while requiring important labour and time resources. On the other hand, data-driven techniques make it possible to build optimal strategies at lower costs. In the field of human-machine dialogue, reinforcement learning has been proven to be particularly useful since no annotation effort is required (unlike supervised techniques) and it is able to learn from delayed rewards (thus making it possible to take the whole dialogue quality as a reward function). In this thesis, a new reinforcement learning turn-taking strategy is proposed and trained using the previous dialogue simulator. In simulation, it has been shown to reduce dialogue duration while improving the task completion ratio when compared to the non-incremental and the handcrafted incremental baselines.

        Finally, the previous strategies have been transposed to a new domain, the Marjordomo, then they have been tested through real users interactions. The handcrafted and the reinforcement learning strategies slightly reduce the dialogue duration, but the latter significantly improves the task completion ratio (by 15\% compared to the non-incremental strategy). Also, compared with the handcrafted strategy, the data-driven one takes more risk by deciding to speak before a user silence (hence significantly reducing the response latency) while maintaining a low false cut-in rate of 6.8\% instead of 31\%.

        To conclude, from a general point of view, this thesis provides new evidence showing the potential of incremental dialogue processing. More particularly, it goes a step further by showing that optimal turn-taking can be learnt automatically during the interactions. Also, the proposed architecture and the generality of the feature used for learning makes it possible to easily transfer this work to any domain and to bootstrap from existing dialogue systems.
        


        
	
	
